%!TeX program = xelatex
\documentclass[zihao=-4,hyperref,a4paper,UTF8,autoindent=2em]{ctexart}

% 引入STY文件,其中已包含基础的包,amsthm,color,graphicx等。具体请查看SWUBachelor.sty
\usepackage{SWUBachelor} 
\ctexset{bibname=\leftline{\zihao{-4}\songti\textbf{参考文献:}}\vspace{-1 cm}}

% 设置封面
\Title{论文标题} % 标题
\Author{你的名字} % 作者名
\Faculty{电子信息工程学院} % 学院名
\Major{信息安全} % 专业名
\Grade{2020} % 年级
\StudentID{222020xxxxxxxxx} % 学号
\Supervisor{xxx} % 导师
\Date{2024}{1}{1} % 日期

%================================正文开始================================%%
\begin{document}

%------------------封面------------------%%
\cover
\thispagestyle{empty} % 首页不显示页码

%------------------目录------------------%%
\newpage
\tableofcontents

%------------------摘要------------------%%
\newpage\setcounter{page}{1} 

% 中文摘要
\chineseabstract
    {这里是中文摘要的内容。在这里写下你的中文摘要。} % 摘要
    {关键词1;关键词2;关键词3} % 关键词

% 英文摘要
\englishabstract
    {English title} % Title
    {Author}  % Author
    {Faculty} % Faculty
    {This is the English abstract. } % Abstract
    {This is the English keywords } % Keywords

%%------------------正文------------------%%
\newpage
\section{模板说明}

本模板主要适用于一些课程的平时论文以及期末论文,默认页边距为2.5cm,中文宋体,英文Times New Roman,字号为12pt(小四)。

编译方式:\verb|xelatex -> bibtex -> xelatex*2|

\subsection{默认模板文件由以下三部分组成:}
\begin{itemize}
    \item \texttt{main.tex} 主文件
    \item \texttt{SWUBachelor.sty} 文档格式控制,包括一些基础的设置,如页眉、标题、姓名等
    \item \texttt{figures} 放置图片的文件夹
\end{itemize}

要更改设置,前往\texttt{SWUBachelor.sty}。

默认带有封面页以及目录页,页码从摘要页开始

\newpage\section{一些插入功能}
\subsection{插入公式}
行内公式$v-\varepsilon+\phi=2$。

插入行间公式如\autoref{Euler}:
\begin{equation}
    v-\varepsilon+\phi=2
    \label{Euler}
\end{equation}

\subsection{插入图片}
SWU校徽如\autoref{swu}所示,注意这里使用了\verb|~\autoref{}|命令,也就是会自动生成“图”“式”等前缀,无需手动输入。

\begin{figure}[!htbp]
    \centering
    \includegraphics[width =.4\textwidth]{figures/swu_logo.png}
    \caption{西南大学}
    \label{swu}
\end{figure}

插入上面图片的代码:

\begin{verbatim}
    \begin{figure}[!htbp]
        \centering
        \includegraphics[width =.4\textwidth]{figures/swu_logo.png}
        \caption{西南大学}
        \label{swu}
    \end{figure}
\end{verbatim}

\subsection{插入文本框}
本模板定义了一个圆角灰底的文本框,使用简化命令\verb|\tbox{}|即可,如果你不喜欢,可以前往 \texttt{SWUBachelor.sty}对其进行修改。

\tbox{
    这是一个圆角灰底的文本框
}

\subsection{插入表格}
本模板文件如\autoref{doc}所示。
\begin{table}[!htbp]
    \centering
    \begin{tabular}{l  | l}
     ine
        文件名 & 说明 \\
         ine
        \texttt{main.tex}  & 主文件 \\
        \texttt{SWUBachelor.sty}  & 文档格式控制\\
        \texttt{figures}  & 图片文件夹 \\
         ine
    \end{tabular}
    \caption{本模板文件组成}
    \label{doc}
\end{table}

%\section{定理环境}
%\begin{Theorem}
%\end{Theorem}
%
%\begin{Lemma}
%\end{Lemma}
%
%\begin{Corollary}
%\end{Corollary}
%
%\begin{Proposition}
%\end{Proposition}
%
%\begin{Definition}
%\end{Definition}
%
%\begin{Example}
%\end{Example}
%
%\begin{proof}
%\end{proof}

\subsection{插入参考文献}
直接使用\verb|\cite{}|即可。

例如:


   \textit{ 此处引用了文献\cite{0Isaac}。此处引用了文献\cite{2016The}}


引用过的文献会自动出现在参考文献中。

\newpage\section{写在最后}
\subsection{发布地址}
\begin{itemize}
    \item Github: \url{https://github.com/LttGenius/SWUBachelor_Latex_Template}
    % \item Overleaf:  \url{https://www.overleaf.com/latex/templates/UCASke-cheng-lun-wen-mo-ban/bcwvxncqffkw}
\end{itemize}

%%----------- 参考文献 -------------------%%

\newpage
\begin{thebibliography}{999}
\addcontentsline{toc}{section}{参考文献}

\bibitem{croosViewDomain}
Tang, J.; Shu, X.; Li, Z.; Jiang, Y.G.; Tian, Q.
\newblock Social Anchor-Unit Graph Regularized Tensor Completion for
  Large-Scale Image Retagging.
\newblock {\em IEEE Trans. Pattern Anal. Mach. Intell.}
  {\bf 2019}, {\em 41},~2027--2034.
\newblock {\url{https://doi.org/10.1109/TPAMI.2019.2906603}}.

\bibitem{classification1}
Han, Z.; Zhang, C.; Fu, H.; Zhou, J.T.
\newblock Trusted Multi-View Classification with Dynamic Evidential Fusion.
\newblock {\em IEEE Trans. Pattern Anal. Mach. Intell.}
  {\bf 2023}, {\em 45},~2551--2566.
\newblock {\url{https://doi.org/10.1109/TPAMI.2022.3171983}}.
\newpage
\end{thebibliography}

%%----------- 致谢 -------------------%%
\acknowledgment{
    这里是致谢
}

%%----------- 附录 -------------------%%
\appendix{
    \section*{附录标题1}
    这里是附录1的内容
    \section*{附录标题2}
    这里是附录2的内容
}

\end{document}